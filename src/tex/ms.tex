%                                                                 aa.dem
% AA vers. 9.1, LaTeX class for Astronomy & Astrophysics
% demonstration file
%                                                       (c) EDP Sciences
%-----------------------------------------------------------------------
%
%\documentclass[referee]{aa} % for a referee version
%\documentclass[onecolumn]{aa} % for a paper on 1 column  
%\documentclass[longauth]{aa} % for the long lists of affiliations
\documentclass[]{aa} % for the letters
%\documentclass[bibyear]{aa} % if the references are not structured
%                              according to the author-year natbib style

%
%\documentclass{aa}  
\usepackage{natbib}
%
\usepackage{color}
\usepackage{hyperref}
\hypersetup{colorlinks=true,allcolors=[rgb]{0,0,0.8}}


\usepackage{showyourwork}

\usepackage{graphicx}
\usepackage{showyourwork}

\usepackage{threeparttable}
%%%%%%%%%%%%%%%%%%%%%%%%%%%%%%%%%%%%%%%%
\usepackage{txfonts}
%%%%%%%%%%%%%%%%%%%%%%%%%%%%%%%%%%%%%%%%
%\usepackage[options]{hyperref}
% To add links in your PDF file, use the package "hyperref"
% with options according to your LaTeX or PDFLaTeX drivers.
%
\begin{document}
   \title{A tilted eclipser around ASASSN-21sa}

   \author{B.C. Ottevanger\inst{1} \and 
   M.A. Kenworthy\inst{1}
          }
   \institute{Leiden Observatory, Leiden University, PO Box 9513, 2300 RA Leiden, The Netherlands \\
              \email{kenworthy@strw.leidenuniv.nl}
             }
             
   \date{Received \today}

% \abstract{}{}{}{}{}
% 5 {} token are mandatory
 
  \abstract
  % context heading (optional)
  {}  
  % {Planets form in disks of gas and dust around stars, and moons are thought to form in circumplanetary disks.
   %
   %There are several systems where a disk around a secondary companion transits in front of the primary star, resulting in a complex and long lived light curve.
   %
   %Understanding the substructure of these disks provides insight into the moon forming process.}
  % aims heading (mandatory)
   {We determine the expected photometric signal for the occultation of background stars by a minor planet in the Oort cloud hosting a Hill sphere ring system.}
  % methods heading (mandatory)
   {We estimate the timescales and number of stars affected by a ring system around an Oort cloud minor planet as it orbits the Sun.}
  % results heading (mandatory)
   {About 1e2 stars will be undergoing dimming/occultation. We can detect most systems beyond XXX au.
   %
   Solid angle subtended by a ring system is independent of distance.
   %
   }
  % conclusions heading (optional), leave it empty if necessary
   {}

   \keywords{giant planet formation --
               moon formation}

   \maketitle
%
%________________________________________________________________

\section{Introduction}\label{sec:intro}

Disks and rings of material have been detected orbiting astrophysical objects from super massive black holes down the mass spectrum to minor planets (Charikolo; REF).

% \begin{figure}[hbt]
% \centering
%     \includegraphics[width=\columnwidth]{figures/parallaxes.pdf}
%     \caption{Plot of different parallactic contributions as a function of distance from the Sun.}
%     \label{fig:parallaxes}
%     \script{plot_parallaxes.py}
% \end{figure}

\section{All sky photometric surveys}\label{sec:surveys}


\begin{acknowledgements}
      This work has made use of data taken from ASAS-SN \citep{shappee_man_2014, kochanek_all-sky_2017}, processed and obtained through \url{https://asas-sn.osu.edu/}.
This research has used the SIMBAD database, operated at CDS, Strasbourg, France \citep{wenger2000}.
%
This work has also made use of data from the European Space Agency (ESA) mission {\it Gaia} (\url{https://www.cosmos.esa.int/gaia}), processed by the {\it Gaia} Data Processing and Analysis Consortium (DPAC, \url{https://www.cosmos.esa.int/web/gaia/dpac/consortium}).
%
Funding for the DPAC has been provided by national institutions, in particular the institutions participating in the {\it Gaia} Multilateral Agreement. Data taken from tic v8.2 was accessed using the VizieR catalogue access tool, CDS, Strasbourg, France.%(DOI : 10.26093/cds/vizier). 
%
The original description of the VizieR service was published in 2000, A\&AS 143, 23 \citep{2000A&AS..143...23O}.
%
This publication makes use of VOSA, developed under the Spanish Virtual Observatory (\url{https://svo.cab.inta-csic.es}) project funded by MCIN/AEI/10.13039/501100011033/ through grant PID2020-112949GB-I00.
%
VOSA has been partially updated by using funding from the European Union's Horizon 2020 Research and Innovation Programme, under Grant Agreement nº 776403 (EXOPLANETS-A)
%
To achieve the scientific results presented in this article we made use of the \emph{Python} programming language\footnote{Python Software Foundation, \url{https://www.python.org/}}, especially the \emph{SciPy} \citep{virtanen2020}, \emph{NumPy} \citep{numpy}, \emph{Matplotlib} \citep{Matplotlib}, \emph{emcee} \citep{foreman-mackey2013}, and \emph{astropy} \citep{astropy_1,astropy_2} packages.
\end{acknowledgements}


\bibliographystyle{aa}
\bibliography{bib}

\end{document}

